\documentclass[11pt]{article}
\usepackage{geometry}
\geometry{margin=1in}
\usepackage{verbatim}

\begin{document}

\section*{Backpropagation \& Gradient-Based Optimization}

\begin{verbatim}
It is basically to teach the network how to change its weights so predictions get closer to targets.
Forward pass is to compute outputs from inputs through layers z = xW + b, a = φ(z).

Loss function is to single number measuring error. The trainer minimizes it.

Backpropagation: uses the chain rule to compute how the loss changes w.r.t each weight (∂L/∂w). That gradient tells us direction to change weights to reduce loss.

Mathematical equivalent

If the loss is L, gradient secrent update for weight w is:

𝑤 ← 𝑤 − 𝜂(∂𝐿/∂𝑤)
where η is the learning rate.

MSE (regression), Binary cross-entropy are among some famous common loss functions.

Optimizers include SGD  which has basic updates which is noisy but simple.

Momentum optimization accumulates velocity to smooth updates.
v ← μv + η∇L and  w ← w - v 

RMSprop scales gradient by running RMS.
s = βs(t−1) + (1−β)gt^2
​w←w−η(gt/root(st+ϵ))

​Adam is an advanced optimizer which combines momentum and adaptive scaling to achieve much better results.

Activation functions include
ReLU is simply max(0,x). It has sparse activations, fast, avoids saturation for positive side.

Sigmoid is smooth, but will saturate for large |x| which results in small gradients.

Softmax is advanced comparatively and produces a probability distribution over classes.


​Vanishing is repeated multiplications by small derivatives (sigmoid/tanh) which makes gradients tiny and results in slow learning and longer process times. The oposite is exploding in which gradients become huge causing parameter blow-up. 
\end{verbatim}

\end{document}
